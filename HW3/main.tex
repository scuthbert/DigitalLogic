\documentclass[12pt]{article}
\setlength{\oddsidemargin}{0in}
\setlength{\evensidemargin}{0in}
\setlength{\textwidth}{6.5in}
\setlength{\parindent}{0in}
\setlength{\parskip}{\baselineskip}

\usepackage[usenames, dvipsnames]{color}
\usepackage{amsmath,amsfonts,amssymb,circuitikz,pdfpages,tikz}

\usetikzlibrary{matrix,calc,circuits.logic.US}
\tikzstyle{branch}=[fill, shape=circle, minimum size=3pt, inner sep=0pt]

%isolated term
%#1 - Optional. Space between node and grouping line. Default=0
%#2 - node
%#3 - filling color
\newcommand{\implicantsol}[3][0]{
    \draw[rounded corners=3pt, fill=#3, opacity=0.3] ($(#2.north west)+(135:#1)$) rectangle ($(#2.south east)+(-45:#1)$);
    }


%internal group
%#1 - Optional. Space between node and grouping line. Default=0
%#2 - top left node
%#3 - bottom right node
%#4 - filling color
\newcommand{\implicant}[4][0]{
    \draw[rounded corners=3pt, fill=#4, opacity=0.3] ($(#2.north west)+(135:#1)$) rectangle ($(#3.south east)+(-45:#1)$);
    }

%group lateral borders
%#1 - Optional. Space between node and grouping line. Default=0
%#2 - top left node
%#3 - bottom right node
%#4 - filling color
\newcommand{\implicantcostats}[4][0]{
    \draw[rounded corners=3pt, fill=#4, opacity=0.3] ($(rf.east |- #2.north)+(90:#1)$)-| ($(#2.east)+(0:#1)$) |- ($(rf.east |- #3.south)+(-90:#1)$);
    \draw[rounded corners=3pt, fill=#4, opacity=0.3] ($(cf.west |- #2.north)+(90:#1)$) -| ($(#3.west)+(180:#1)$) |- ($(cf.west |- #3.south)+(-90:#1)$);
}

%group top-bottom borders
%#1 - Optional. Space between node and grouping line. Default=0
%#2 - top left node
%#3 - bottom right node
%#4 - filling color
\newcommand{\implicantdaltbaix}[4][0]{
    \draw[rounded corners=3pt, fill=#4, opacity=0.3] ($(cf.south -| #2.west)+(180:#1)$) |- ($(#2.south)+(-90:#1)$) -| ($(cf.south -| #3.east)+(0:#1)$);
    \draw[rounded corners=3pt, fill=#4, opacity=0.3] ($(rf.north -| #2.west)+(180:#1)$) |- ($(#3.north)+(90:#1)$) -| ($(rf.north -| #3.east)+(0:#1)$);
}

%group corners
%#1 - Optional. Space between node and grouping line. Default=0
%#2 - filling color
\newcommand{\implicantcantons}[2][0]{
    \draw[rounded corners=3pt, opacity=.3] ($(rf.east |- 0.south)+(-90:#1)$) -| ($(0.east |- cf.south)+(0:#1)$);
    \draw[rounded corners=3pt, opacity=.3] ($(rf.east |- 8.north)+(90:#1)$) -| ($(8.east |- rf.north)+(0:#1)$);
    \draw[rounded corners=3pt, opacity=.3] ($(cf.west |- 2.south)+(-90:#1)$) -| ($(2.west |- cf.south)+(180:#1)$);
    \draw[rounded corners=3pt, opacity=.3] ($(cf.west |- 10.north)+(90:#1)$) -| ($(10.west |- rf.north)+(180:#1)$);
    \fill[rounded corners=3pt, fill=#2, opacity=.3] ($(rf.east |- 0.south)+(-90:#1)$) -|  ($(0.east |- cf.south)+(0:#1)$) [sharp corners] ($(rf.east |- 0.south)+(-90:#1)$) |-  ($(0.east |- cf.south)+(0:#1)$) ;
    \fill[rounded corners=3pt, fill=#2, opacity=.3] ($(rf.east |- 8.north)+(90:#1)$) -| ($(8.east |- rf.north)+(0:#1)$) [sharp corners] ($(rf.east |- 8.north)+(90:#1)$) |- ($(8.east |- rf.north)+(0:#1)$) ;
    \fill[rounded corners=3pt, fill=#2, opacity=.3] ($(cf.west |- 2.south)+(-90:#1)$) -| ($(2.west |- cf.south)+(180:#1)$) [sharp corners]($(cf.west |- 2.south)+(-90:#1)$) |- ($(2.west |- cf.south)+(180:#1)$) ;
    \fill[rounded corners=3pt, fill=#2, opacity=.3] ($(cf.west |- 10.north)+(90:#1)$) -| ($(10.west |- rf.north)+(180:#1)$) [sharp corners] ($(cf.west |- 10.north)+(90:#1)$) |- ($(10.west |- rf.north)+(180:#1)$) ;
}

%Empty Karnaugh map 4x4
\newenvironment{Karnaugh}%
{
\begin{tikzpicture}[baseline=(current bounding box.north),scale=0.8]
\draw (0,0) grid (4,4);
\draw (0,4) -- node [pos=1,above right,anchor=south west] {$x_3x_4$} node [pos=0.7,below left,anchor=north east] {$x_1x_2$} ++(135:1);
%
\matrix (mapa) [matrix of nodes,
        column sep={0.8cm,between origins},
        row sep={0.8cm,between origins},
        every node/.style={minimum size=0.3mm},
        anchor=8.center,
        ampersand replacement=\&] at (0.5,0.5)
{
                       \& |(c00)| 00         \& |(c01)| 01         \& |(c11)| 11         \& |(c10)| 10         \& |(cf)| \phantom{00} \\
|(r00)| 00             \& |(0)|  \phantom{0} \& |(1)|  \phantom{0} \& |(3)|  \phantom{0} \& |(2)|  \phantom{0} \&                     \\
|(r01)| 01             \& |(4)|  \phantom{0} \& |(5)|  \phantom{0} \& |(7)|  \phantom{0} \& |(6)|  \phantom{0} \&                     \\
|(r11)| 11             \& |(12)| \phantom{0} \& |(13)| \phantom{0} \& |(15)| \phantom{0} \& |(14)| \phantom{0} \&                     \\
|(r10)| 10             \& |(8)|  \phantom{0} \& |(9)|  \phantom{0} \& |(11)| \phantom{0} \& |(10)| \phantom{0} \&                     \\
|(rf) | \phantom{00}   \&                    \&                    \&                    \&                    \&                     \\
};
}%
{
\end{tikzpicture}
}

%Empty Karnaugh map 2x4
\newenvironment{Karnaughvuit}%
{
\begin{tikzpicture}[baseline=(current bounding box.north),scale=0.8]
\draw (0,0) grid (4,2);
\draw (0,2) -- node [pos=1,above right,anchor=south west] {$x_2x_3$} node [pos=0.7,below left,anchor=north east] {$x_1$} ++(135:1);
%
\matrix (mapa) [matrix of nodes,
        column sep={0.8cm,between origins},
        row sep={0.8cm,between origins},
        every node/.style={minimum size=0.3mm},
        anchor=4.center,
        ampersand replacement=\&] at (0.5,0.5)
{
                      \& |(c00)| 00         \& |(c01)| 01         \& |(c11)| 11         \& |(c10)| 10         \& |(cf)| \phantom{00} \\
|(r00)| 0             \& |(0)|  \phantom{0} \& |(1)|  \phantom{0} \& |(3)|  \phantom{0} \& |(2)|  \phantom{0} \&                     \\
|(r01)| 1             \& |(4)|  \phantom{0} \& |(5)|  \phantom{0} \& |(7)|  \phantom{0} \& |(6)|  \phantom{0} \&                     \\
|(rf) | \phantom{00}  \&                    \&                    \&                    \&                    \&                     \\
};
}%
{
\end{tikzpicture}
}

%Empty Karnaugh map 2x2
\newenvironment{Karnaughquatre}%
{
\begin{tikzpicture}[baseline=(current bounding box.north),scale=0.8]
\draw (0,0) grid (2,2);
\draw (0,2) -- node [pos=0.7,above right,anchor=south west] {b} node [pos=0.7,below left,anchor=north east] {a} ++(135:1);
%
\matrix (mapa) [matrix of nodes,
        column sep={0.8cm,between origins},
        row sep={0.8cm,between origins},
        every node/.style={minimum size=0.3mm},
        anchor=2.center,
        ampersand replacement=\&] at (0.5,0.5)
{
          \& |(c00)| 0          \& |(c01)| 1  \\
|(r00)| 0 \& |(0)|  \phantom{0} \& |(1)|  \phantom{0} \\
|(r01)| 1 \& |(2)|  \phantom{0} \& |(3)|  \phantom{0} \\
};
}%
{
\end{tikzpicture}
}

%Defines 8 or 16 values (0,1,X)
\newcommand{\contingut}[1]{%
\foreach \x [count=\xi from 0]  in {#1}
     \path (\xi) node {\x};
}

%Places 1 in listed positions
\newcommand{\minterms}[1]{%
    \foreach \x in {#1}
        \path (\x) node {1};
}

%Places 0 in listed positions
\newcommand{\maxterms}[1]{%
    \foreach \x in {#1}
        \path (\x) node {0};
}

\newcommand{\overbar}[1]{\mkern 1.5mu\overline{\mkern-1.5mu#1\mkern-1.5mu}\mkern 1.5mu}

\begin{document}
\title{Digital Logic Homework 3}

ECEN 2350 Spring 2017 \hfill Homework 3\\
Samuel Cuthbertson

\hrulefill

\begin{enumerate}
	\vspace{-4mm}
	\item \textit{Book Problems: 2.37, 2.39 and 2.46}
	\begin{enumerate}
    	
		\item[(2.37)] \textit{Find the minimum cost SOP and POS forms for $f(x_1,x_2,x_3)=\sum m(1,2,3,5)$}
    	\begin{center}
			\begin{Karnaughvuit}
       	 		\contingut{0,1,1,1,0,1,0,0}
       			\implicant{3}{2}{red}
       			\implicant{1}{5}{orange}
       			\implicant{0}{4}{green}
       			\implicant{7}{6}{cyan}
    		\end{Karnaughvuit}
    	\end{center}
        Therefore, we have the simplified (minimum cost) SOP and POS equations:
        \begin{align*}
        SOP &= \color{red}{\overbar{x_1}x_2} \color{black}{+} \color{orange}{\overbar{x_2}x_3} \\
        POS &= \color{cyan}{(\overbar{x_1}+\overbar{x_2})}\color{green}{(x_2+x_3)}
        \end{align*}
        
        
    	\item[(2.39)] \textit{Find the minimum cost SOP and POS forms for} $f(x_1,x_2,x_3,x_4)=$\\$= \prod M(0,1,2,4,5,7,8,9,10,12,14,15)$
		\begin{center}
			\begin{Karnaugh}
       	 		\contingut{0,0,0,1,0,0,1,0,0,0,0,1,0,1,0,0}
       			\implicantdaltbaix{3}{11}{yellow}
       			\implicantsol{6}{orange}
       			\implicantsol{13}{red}
                \implicantdaltbaix{0}{9}{blue}
                \implicant{0}{8}{teal}
                \implicant{5}{7}{cyan}
                \implicant{15}{14}{green}
                \implicantcantons[2pt]{magenta}
    		\end{Karnaugh}
    	\end{center}
        Therefore, we have the simplified (minimum cost) SOP and POS equations:
        \begin{align*}
        SOP &= \color{red}{x_1x_2\overbar{x_3}x_4} \color{black}{+} \color{Maroon}{\overbar{x_2}x_3x_4} \color{black}{+} \color{orange}{\overbar{x_1}x_2x_3\overbar{x_4}} \\
        POS &= \color{blue}{(x_2+x_3)}\color{cyan}{(x_1+\overbar{x_2}+\overbar{x_4})}\color{green}{(\overbar{x_1}+\overbar{x_2}+\overbar{x_3})}\color{teal}{(x_3+x_4)}\color{magenta}{(x_2+x_4)}
        \end{align*}

		\newpage
    	\item[(2.46)] \textit{Derive a minimum cost realization of the four-variable function that is equal to 1 if exactly two or exactly three of its variables are equal to 1; otherwise it is equal to 0.}
		\begin{center}
			\begin{Karnaugh}
       	 		\contingut{0,0,0,1,0,1,1,1,0,1,1,1,1,1,1,0}
                \implicant{0}{1}{blue}
                \implicant{0}{4}{cyan}
                \implicantcostats{0}{2}{green}
                \implicantsol{15}{teal}
                \implicantdaltbaix{0}{8}{violet}
    		\end{Karnaugh}
    	\end{center}
        As there are fewer $0$'s than $1$'s ($5 < 10$), the lowest cost formula will be the POS using maxterms, as shown here:
        \begin{align*}
        POS &= \color{blue}{(x_1+x_2+x_3)}\color{cyan}{(x_1+x_3+x_4)}\color{green}{(x_1+x_2+x_4)}\color{teal}{(\overbar{x_1}+\overbar{x_2}+\overbar{x_3}+\overbar{x_4})}\color{violet}{(x_2+x_3+x_4)}
        \end{align*}
        \begin{center}
        Also, note that $\color{teal}{(\overbar{x_1}+\overbar{x_2}+\overbar{x_3}+\overbar{x_4})} = \color{teal}{\overbar{(x_1x_2x_3x_4)}}$ 
        \end{center}
		\begin{tikzpicture}[circuit logic US,scale=1.5]
  			\node (x1) at (0, 3) {$x_1$};
    		\node (x2) at (0, 2) {$x_2$};
    		\node (x3) at (0, 1) {$x_3$};
    		\node (x4) at (0, 0) {$x_4$};

    		\node[or gate US, draw, rotate=0, logic gate inputs=nnn, draw=blue] at (2.5,2.5) (123) {};
    		\node[or gate US, draw, rotate=0, logic gate inputs=nnn, draw=cyan] at (2.5,.5) (234) {};
    		\node[or gate US, draw, rotate=0, logic gate inputs=nnn, draw=violet] at (2.5,-.5) (134) {};
    		\node[or gate US, draw, rotate=0, logic gate inputs=nnn, draw=green] at (2.5,3.5) (124) {};
    		\node[nand gate US, draw, rotate=0, logic gate inputs=nnn, draw=teal] at (2.5,1.5) (1234) {};
    		\node[and gate US, draw, rotate=0, logic gate inputs=nnnnn, draw=black] at (5,1.5) (124) {};          
		\end{tikzpicture}
	\end{enumerate}
	
    \newpage
    \item \textit{Find the logic equation for the lowest cost circuit for the implementation below in SOP, and then synthesize the circuit with only NAND gates.}
    \[
    f(x_1,x_2,x_3,x_4) = \sum m(0,4,5,6,7,14)
    \]
    \begin{center}
			\begin{Karnaugh}
       	 		\contingut{1,0,0,0,1,1,1,1,0,0,0,0,0,0,1,0}
                \implicant{0}{4}{red}
                \implicant{4}{6}{orange}
                \implicant{6}{14}{yellow}
            \end{Karnaugh}
    \end{center}
	\begin{align*}
        SOP &= \color{red}{\overbar{x_1}\overbar{x_3}\overbar{x_4}}\color{black}{+}\color{orange}{\overbar{x_1}x_2}\color{black}{+}\color{Maroon}{x_2x_3\overbar{x_4}} 
    \end{align*}
		\begin{tikzpicture}[circuit logic US,scale=1.5]
  			\node (x1) at (0, 3) {$x_1$};
    		\node (x2) at (0, 2) {$x_2$};
    		\node (x3) at (0, 1) {$x_3$};
    		\node (x4) at (0, 0) {$x_4$};
			\node[nand gate US, draw, rotate=0, logic gate inputs=nnn, draw=black] at (1.5,3) (not1) {};
            \node[nand gate US, draw, rotate=0, logic gate inputs=nnn, draw=black] at (1.5,1) (not3) {};
            \node[nand gate US, draw, rotate=0, logic gate inputs=nnn, draw=black] at (1.5,0) (not4) {};
			
    		\node[nand gate US, draw, rotate=0, logic gate inputs=nnn, draw=yellow] at (3.5,0.5) (234) {};
    		\node[nand gate US, draw, rotate=0, logic gate inputs=nnn, draw=orange] at (3.5,1.5) (12) {};
    		\node[nand gate US, draw, rotate=0, logic gate inputs=nnn, draw=red] at (3.5,2.5) (134) {};
    		\node[nand gate US, draw, rotate=0, logic gate inputs=nnn, draw=black] at (5.5,1.5) (final) {};
            
		\end{tikzpicture}
    
    \newpage
    \item \textit{Find the logic equation for the lowest cost circuit for the implementation below in POS, and then synthesize the circuit with only NOR gates.}
    \[
    f(x_1,x_2,x_3,x_4) = \prod M(0,2,6,7,8,10,12,13,14,15)
    \]
    \begin{center}
			\begin{Karnaugh}
       	 		\contingut{0,1,0,1,1,1,0,0,0,1,0,1,0,0,0,0}
                \implicantcantons[2pt]{blue}
                \implicant{7}{14}{cyan}
                \implicant{12}{14}{green}
            \end{Karnaugh}
    \end{center}
	\begin{align*}
        POS &= \color{blue}{(x_2+x_4)}\color{cyan}{(\overbar{x_2}+\overbar{x_3})}\color{green}{(\overbar{x_1}+\overbar{x_2})} 
    \end{align*}
	\begin{tikzpicture}[circuit logic US,scale=1.5]
  			\node (x1) at (0, 3) {$x_1$};
    		\node (x2) at (0, 2) {$x_2$};
    		\node (x3) at (0, 1) {$x_3$};
    		\node (x4) at (0, 0) {$x_4$};
			\node[nor gate US, draw, rotate=0, logic gate inputs=nnn, draw=black] at (1.5,3) (not1) {};
            \node[nor gate US, draw, rotate=0, logic gate inputs=nnn, draw=black] at (1.5,2) (not2) {};
            \node[nor gate US, draw, rotate=0, logic gate inputs=nnn, draw=black] at (1.5,1) (not3) {};
			
    		\node[nor gate US, draw, rotate=0, logic gate inputs=nnn, draw=green] at (3.5,0.5) (12) {};
    		\node[nor gate US, draw, rotate=0, logic gate inputs=nnn, draw=cyan] at (3.5,1.5) (23) {};
    		\node[nor gate US, draw, rotate=0, logic gate inputs=nnn, draw=blue] at (3.5,2.5) (24) {};
    		\node[nor gate US, draw, rotate=0, logic gate inputs=nnn, draw=black] at (5.5,1.5) (final) {};
            
		\end{tikzpicture}
    
    \newpage
    \item \textit{Given the following truth table (from $HW1$,$HW2$) use a Karnaugh map to:}
    \begin{center}
    \hfill
    \begin{minipage}{0.2\textwidth}
    \begin{center}
    	\begin{tabular}{ccc|c}
    		$x_1$ & $x_2$ & $x_3$ & f \\
    		\hline
    		0 & 0 & 0 & 0 \\
    		0 & 0 & 1 & 1 \\
    		0 & 1 & 0 & 0 \\
    		0 & 1 & 1 & 1 \\
    		1 & 0 & 0 & 1 \\
    		1 & 0 & 1 & 0 \\
    		1 & 1 & 0 & 1 \\
    		1 & 1 & 1 & 1 \\
    	\end{tabular}
    \end{center}
    \end{minipage}
    \hfill
  	\begin{minipage}{0.5\textwidth}
    \begin{center}
  		\begin{Karnaughvuit}
       		\contingut{0,1,0,1,1,0,1,1}
  		\end{Karnaughvuit}
    \end{center}
	\end{minipage}
    \end{center}
    \begin{enumerate}
    \item \textit{Show that there are two minimized solutions for SOP form.}
    
    Both minimized forms are shown below, as there are two ways to select all the one's from the K-Map with a minimum number of selections.
    \begin{center}
   	\begin{minipage}{0.4\textwidth}
    \begin{center}
  		\begin{Karnaughvuit}
       		\contingut{0,1,0,1,1,0,1,1}
            \implicant{1}{3}{red}
            \implicant{7}{6}{orange}
            \implicantcostats{4}{6}{yellow}
  		\end{Karnaughvuit}
        \[
        f(x_1,x_2,x_3) = \color{red}{\overbar{x_1}x_3}\color{black}{+}\color{orange}{x_1x_2}\color{black}{+}\color{Maroon}{x_1\overbar{x_3}}
        \]
    \end{center}
    \end{minipage}
    \hfill
  	\begin{minipage}{0.4\textwidth}
    \begin{center}
  		\begin{Karnaughvuit}
       		\contingut{0,1,0,1,1,0,1,1}
            \implicant{1}{3}{red}
            \implicant{3}{7}{orange}
            \implicantcostats{4}{6}{yellow}
  		\end{Karnaughvuit}
        \[
        f(x_1,x_2,x_3) = \color{red}{\overbar{x_1}x_3}\color{black}{+}\color{orange}{x_2x_3}\color{black}{+}\color{Maroon}{x_1\overbar{x_3}}
        \]
    \end{center}
	\end{minipage}
    \end{center}
    
    \item \textit{Show the minimized POS form.}
        \begin{center}
  		\begin{Karnaughvuit}
       		\contingut{0,1,0,1,1,0,1,1}
            \implicantcostats{0}{2}{blue}
            \implicantsol{5}{green}
  		\end{Karnaughvuit}
        \[
        f(x_1,x_2,x_3) = \color{blue}{(x_1+x_3)}\color{green}{(\overbar{x_1}+x_2+\overbar{x_3})}
        \]
    \end{center}
    \end{enumerate}
    
    
\end{enumerate}
\end{document}