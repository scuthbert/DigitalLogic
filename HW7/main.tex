\documentclass[12pt]{article}
\setlength{\oddsidemargin}{0in}
\setlength{\evensidemargin}{0in}
\setlength{\textwidth}{6.5in}
\setlength{\parindent}{0in}
\setlength{\parskip}{\baselineskip}

\usepackage[letterpaper, portrait, margin=1in]{geometry}
\usepackage[usenames, dvipsnames, rgb]{xcolor}
\usepackage{amsmath,amsfonts,amssymb,circuitikz,pdfpages,tikz,fancyvrb}
\usepackage{tikz-timing}

\usetikzlibrary{matrix,calc,circuits.logic.US, shapes.geometric}
\tikzstyle{branch}=[fill, shape=circle, minimum size=3pt, inner sep=0pt]

\newcommand{\overbar}[1]
    {\mkern1.5mu\overline{\mkern-1.5mu#1\mkern-1.5mu}\mkern1.5mu}

\begin{document}
\title{Digital Logic Homework 4}

ECEN 2350 Spring 2017 \hfill Homework 4\\
Samuel Cuthbertson

\hrulefill{}
\begin{enumerate}
  \item \textit{Book Problems: 5.7, 5.10, 5.16, 5.25}
  \begin{enumerate}
    \item[5.7:] \textit{Show how a JK flip-flop can be implemented with a T
                        flip-flop and other gates.}

      Placeholder answer

    \item[5.10:] \textit{Write the behavioral Verilog code for a JK flip-flop.}

      Placeholder answer

    \item[5.16:] \textit{Design a three bit up/down counter using T flip-flops,
                        using a control signal of $\overbar{UP}/DOWN$.}

      Placeholder answer

    \item[5.25:] \textit{Using the circuit shown in the book, complete the
                         below timing diagram.}

      Placeholder answer

  \end{enumerate}

  \item \textit{Draw a 4-bit Universal shift register that can parallel load,
                shift left, shift right, and synchronously clear. Provide a
                characteristic table for the control operations.}

    Placeholder answer

  \item \textit{Implement a 3-bit up-counter using only one D flip-flop, one
                T flip-flop, one JK flip-flop and one AND gate. Assume all
                flip-flops are positive edge triggered. Show your circuit and
                a timing diagram.}

    Placeholder answer

  \item \textit{Write the behavioural Verilog for 5.16, calling the module
                \texttt{up\_down\_counter}.}

    Placeholder answer
\end{enumerate}
\end{document}
