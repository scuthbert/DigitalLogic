\documentclass[12pt]{article}
\setlength{\oddsidemargin}{0in}
\setlength{\evensidemargin}{0in}
\setlength{\textwidth}{6.5in}
\setlength{\parindent}{0in}
\setlength{\parskip}{\baselineskip}

\usepackage[letterpaper, portrait, margin=1in]{geometry}
\usepackage[usenames, dvipsnames, rgb]{xcolor}
\usepackage{amsmath,amsfonts,amssymb,circuitikz,pdfpages,tikz,fancyvrb}
\usepackage{tikz-timing}

\usetikzlibrary{matrix,calc,circuits.logic.US,shapes.geometric}
\tikzstyle{branch}=[fill, shape=circle, minimum size=3pt, inner sep=0pt]
\tikzstyle{mux} = [ trapezium,   draw,
                    shape border rotate = 270, trapezium angle = 60,
                    inner ysep=0pt, outer sep=1pt, inner xsep=1pt,
                    text width = 1.5em, align = center,
                    node distance=3cm]


\newcommand{\overbar}[1]
    {\mkern1.5mu\overline{\mkern-1.5mu#1\mkern-1.5mu}\mkern1.5mu}

\begin{document}

\renewcommand{\arraystretch}{1.25}

\makeatletter
    \pgfdeclareshape{JK}{
      \savedanchor\northeast{%
        \pgfmathsetlength\pgf@x{\pgfshapeminwidth}%
        \pgfmathsetlength\pgf@y{\pgfshapeminheight}%
        \pgf@x=0.5\pgf@x
        \pgf@y=0.5\pgf@y
      }
      % This is redundant, but makes some things easier:
      \savedanchor\southwest{%
        \pgfmathsetlength\pgf@x{\pgfshapeminwidth}%
        \pgfmathsetlength\pgf@y{\pgfshapeminheight}%
        \pgf@x=-0.5\pgf@x
        \pgf@y=-0.5\pgf@y
      }
      \inheritsavedanchors[from=rectangle]
      \inheritanchorborder[from=rectangle]
      \anchor{center}{\pgfpointorigin}
      \anchor{north}{\northeast \pgf@x=0pt}
      \anchor{east}{\northeast \pgf@y=0pt}
      \anchor{south}{\southwest \pgf@x=0pt}
      \anchor{west}{\southwest \pgf@y=0pt}
      \anchor{north east}{\northeast}
      \anchor{north west}{\northeast \pgf@x=-\pgf@x}
      \anchor{south west}{\southwest}
      \anchor{south east}{\southwest \pgf@x=-\pgf@x}
      \anchor{J}{
        \pgf@process{\northeast}%
        \pgf@x=-1\pgf@x%
        \pgf@y=.5\pgf@y%
      }
      \anchor{CLK}{
        \pgf@process{\northeast}%
        \pgf@x=-1\pgf@x%
        \pgf@y=0\pgf@y%
      }
      \anchor{K}{
        \pgf@process{\northeast}%
        \pgf@x=-1\pgf@x%
        \pgf@y=-0.5\pgf@y%
      }
      \anchor{Q}{
        \pgf@process{\northeast}%
        \pgf@y=.5\pgf@y%
      }
      \anchor{Qn}{
        \pgf@process{\northeast}%
        \pgf@y=-.5\pgf@y%
      }

      \backgroundpath{
        % Rectangle box
        \pgfpathrectanglecorners{\southwest}{\northeast}
        % Angle (>) for clock input
        \pgf@anchor@JK@CLK
        \pgf@xa=\pgf@x \pgf@ya=\pgf@y
        \pgf@xb=\pgf@x \pgf@yb=\pgf@y
        \pgf@xc=\pgf@x \pgf@yc=\pgf@y
        \pgfmathsetlength\pgf@x{1.6ex} % size depends on font size
        \advance\pgf@ya by \pgf@x
        \advance\pgf@xb by \pgf@x
        \advance\pgf@yc by -\pgf@x
        \pgfpathmoveto{\pgfpoint{\pgf@xa}{\pgf@ya}}
        \pgfpathlineto{\pgfpoint{\pgf@xb}{\pgf@yb}}
        \pgfpathlineto{\pgfpoint{\pgf@xc}{\pgf@yc}}
        \pgfclosepath

        \begingroup

        \pgf@anchor@JK@K
        \pgftext[left,base,at={\pgfpoint{\pgf@x}{\pgf@y}},
          x=\pgfshapeinnerxsep]{\raisebox{-0.75ex}{K}}


        \pgf@anchor@JK@J
        \pgftext[left,base,at={\pgfpoint{\pgf@x}{\pgf@y}},
          x=\pgfshapeinnerxsep]{\raisebox{-0.75ex}{J}}

        \pgf@anchor@JK@Q
        \pgftext[right,base,at={\pgfpoint{\pgf@x}{\pgf@y}},
          x=-\pgfshapeinnerxsep]{\raisebox{-.75ex}{Q}}

        \pgf@anchor@JK@Qn
        \pgftext[right,base,at={\pgfpoint{\pgf@x}{\pgf@y}},
          x=-\pgfshapeinnerxsep]{\raisebox{-.75ex}{$\overline{\mbox{Q}}$}}

        \endgroup
      }
    }
    % Key to add font macros to the current font
    \tikzset{add font/.code={\expandafter\def\expandafter\tikz@textfont
            \expandafter{\tikz@textfont#1}}}

    % Define default style for this node
    \tikzset{flip flop/port labels/.style={font=\sffamily\scriptsize}}
    \tikzset{every JK node/.style={draw,minimum width=1.5cm,
            minimum height=2.25cm,inner sep=1mm,outer sep=0pt,
            cap=round,add font=\sffamily}}
\makeatother

\makeatletter
    \pgfdeclareshape{tff}{
      \savedanchor\northeast{%
        \pgfmathsetlength\pgf@x{\pgfshapeminwidth}%
        \pgfmathsetlength\pgf@y{\pgfshapeminheight}%
        \pgf@x=0.5\pgf@x
        \pgf@y=0.5\pgf@y
      }
      % This is redundant, but makes some things easier:
      \savedanchor\southwest{%
        \pgfmathsetlength\pgf@x{\pgfshapeminwidth}%
        \pgfmathsetlength\pgf@y{\pgfshapeminheight}%
        \pgf@x=-0.5\pgf@x
        \pgf@y=-0.5\pgf@y
      }
      \inheritsavedanchors[from=rectangle]
      \inheritanchorborder[from=rectangle]
      \anchor{center}{\pgfpointorigin}
      \anchor{north}{\northeast \pgf@x=0pt}
      \anchor{east}{\northeast \pgf@y=0pt}
      \anchor{south}{\southwest \pgf@x=0pt}
      \anchor{west}{\southwest \pgf@y=0pt}
      \anchor{north east}{\northeast}
      \anchor{north west}{\northeast \pgf@x=-\pgf@x}
      \anchor{south west}{\southwest}
      \anchor{south east}{\southwest \pgf@x=-\pgf@x}
      \anchor{T}{
        \pgf@process{\northeast}%
        \pgf@x=-1\pgf@x%
        \pgf@y=.5\pgf@y%
      }
      \anchor{CLK}{
        \pgf@process{\northeast}%
        \pgf@x=-1\pgf@x%
        \pgf@y=-.5\pgf@y%
      }
      \anchor{Q}{
        \pgf@process{\northeast}%
        \pgf@y=.5\pgf@y%
      }
      \anchor{Qn}{
        \pgf@process{\northeast}%
        \pgf@y=-.5\pgf@y%
      }

      \backgroundpath{
        % Rectangle box
        \pgfpathrectanglecorners{\southwest}{\northeast}
        % Angle (>) for clock input
        \pgf@anchor@tff@CLK
        \pgf@xa=\pgf@x \pgf@ya=\pgf@y
        \pgf@xb=\pgf@x \pgf@yb=\pgf@y
        \pgf@xc=\pgf@x \pgf@yc=\pgf@y
        \pgfmathsetlength\pgf@x{1.6ex} % size depends on font size
        \advance\pgf@ya by \pgf@x
        \advance\pgf@xb by \pgf@x
        \advance\pgf@yc by -\pgf@x
        \pgfpathmoveto{\pgfpoint{\pgf@xa}{\pgf@ya}}
        \pgfpathlineto{\pgfpoint{\pgf@xb}{\pgf@yb}}
        \pgfpathlineto{\pgfpoint{\pgf@xc}{\pgf@yc}}
        \pgfclosepath

        \begingroup

        \pgf@anchor@tff@T
        \pgftext[left,base,at={\pgfpoint{\pgf@x}{\pgf@y}},
          x=\pgfshapeinnerxsep]{\raisebox{-0.75ex}{T}}

        \pgf@anchor@tff@Q
        \pgftext[right,base,at={\pgfpoint{\pgf@x}{\pgf@y}},
          x=-\pgfshapeinnerxsep]{\raisebox{-.75ex}{Q}}

        \pgf@anchor@tff@Qn
        \pgftext[right,base,at={\pgfpoint{\pgf@x}{\pgf@y}},
          x=-\pgfshapeinnerxsep]{\raisebox{-.75ex}{$\overline{\mbox{Q}}$}}

        \endgroup
      }
    }
    % Key to add font macros to the current font
    \tikzset{add font/.code={\expandafter\def\expandafter\tikz@textfont
            \expandafter{\tikz@textfont#1}}}

    % Define default style for this node
    \tikzset{flip flop/port labels/.style={font=\sffamily\scriptsize}}
    \tikzset{every tff node/.style={draw,minimum width=1.5cm,
            minimum height=2.25cm,inner sep=1mm,outer sep=0pt,
            cap=round,add font=\sffamily}}
\makeatother

\makeatletter
    \pgfdeclareshape{dff}{
      \savedanchor\northeast{%
        \pgfmathsetlength\pgf@x{\pgfshapeminwidth}%
        \pgfmathsetlength\pgf@y{\pgfshapeminheight}%
        \pgf@x=0.5\pgf@x
        \pgf@y=0.5\pgf@y
      }
      % This is redundant, but makes some things easier:
      \savedanchor\southwest{%
        \pgfmathsetlength\pgf@x{\pgfshapeminwidth}%
        \pgfmathsetlength\pgf@y{\pgfshapeminheight}%
        \pgf@x=-0.5\pgf@x
        \pgf@y=-0.5\pgf@y
      }
      \inheritsavedanchors[from=rectangle]
      \inheritanchorborder[from=rectangle]
      \anchor{center}{\pgfpointorigin}
      \anchor{north}{\northeast \pgf@x=0pt}
      \anchor{east}{\northeast \pgf@y=0pt}
      \anchor{south}{\southwest \pgf@x=0pt}
      \anchor{west}{\southwest \pgf@y=0pt}
      \anchor{north east}{\northeast}
      \anchor{north west}{\northeast \pgf@x=-\pgf@x}
      \anchor{south west}{\southwest}
      \anchor{south east}{\southwest \pgf@x=-\pgf@x}
      \anchor{D}{
        \pgf@process{\northeast}%
        \pgf@x=-1\pgf@x%
        \pgf@y=.5\pgf@y%
      }
      \anchor{CLK}{
        \pgf@process{\northeast}%
        \pgf@x=-1\pgf@x%
        \pgf@y=-.5\pgf@y%
      }
      \anchor{Q}{
        \pgf@process{\northeast}%
        \pgf@y=.5\pgf@y%
      }
      \anchor{Qn}{
        \pgf@process{\northeast}%
        \pgf@y=-.5\pgf@y%
      }

      \backgroundpath{
        % Rectangle box
        \pgfpathrectanglecorners{\southwest}{\northeast}
        % Angle (>) for clock input
        \pgf@anchor@dff@CLK
        \pgf@xa=\pgf@x \pgf@ya=\pgf@y
        \pgf@xb=\pgf@x \pgf@yb=\pgf@y
        \pgf@xc=\pgf@x \pgf@yc=\pgf@y
        \pgfmathsetlength\pgf@x{1.6ex} % size depends on font size
        \advance\pgf@ya by \pgf@x
        \advance\pgf@xb by \pgf@x
        \advance\pgf@yc by -\pgf@x
        \pgfpathmoveto{\pgfpoint{\pgf@xa}{\pgf@ya}}
        \pgfpathlineto{\pgfpoint{\pgf@xb}{\pgf@yb}}
        \pgfpathlineto{\pgfpoint{\pgf@xc}{\pgf@yc}}
        \pgfclosepath

        \begingroup

        \pgf@anchor@dff@D
        \pgftext[left,base,at={\pgfpoint{\pgf@x}{\pgf@y}},
          x=\pgfshapeinnerxsep]{\raisebox{-0.75ex}{D}}

        \pgf@anchor@dff@Q
        \pgftext[right,base,at={\pgfpoint{\pgf@x}{\pgf@y}},
          x=-\pgfshapeinnerxsep]{\raisebox{-.75ex}{Q}}

        \pgf@anchor@dff@Qn
        \pgftext[right,base,at={\pgfpoint{\pgf@x}{\pgf@y}},
          x=-\pgfshapeinnerxsep]{\raisebox{-.75ex}{$\overline{\mbox{Q}}$}}

        \endgroup
      }
    }
    % Key to add font macros to the current font
    \tikzset{add font/.code={\expandafter\def\expandafter\tikz@textfont
            \expandafter{\tikz@textfont#1}}}

    % Define default style for this node
    \tikzset{flip flop/port labels/.style={font=\sffamily\scriptsize}}
    \tikzset{every dff node/.style={draw,minimum width=1.5cm,
            minimum height=2.25cm,inner sep=1mm,outer sep=0pt,
            cap=round,add font=\sffamily}}
\makeatother

\title{Digital Logic Homework 8}

ECEN 2350 Spring 2017 \hfill Homework 8\\
Samuel Cuthbertson

\hrulefill{}
\begin{enumerate}
  \item \textit{Book Problems: 6.9, 6.11}
  \begin{enumerate}
    \item[6.9:] \textit{A sequential circuit has two inputs, $w_1$ and $w_2$, and an output $z$. Its function is to compare the input sequences on the two inputs. If $w_1 = w_2$ during any four consecutive clock cycles, the circuit produces $z=1$; otherwise, $z=0$. Derive a suitable circuit.}

    \item[6.11:] \textit{A given FSM has an input, $w$, and an output, $z$. During four consecutive clock pulses, a sequence of four values of the $w$ signal is applied. Derive a state table for the FSM that produces $z=1$ when it detects that either the sequence $w:0010$ or $w:1110$ has been applied; otherwise, $z=0$. After the fourth clock pulse, the machince has to be again in the reset state, ready for the next sequence. Minimize the number of states needed.}

  \end{enumerate}

  \newpage
  \item \textit{Book Problem: B.2}
  \begin{enumerate}
    \item[a:] \textit{Show that the circuit in Figure PB.2 is fuctionally equivalent to the circuit in Figure PB.1.}



    \item[b:] \textit{How many transistors are needed to build this CMOS circuit? Draw the CMOS circuit.}
  \end{enumerate}

  \newpage
  \item \textit{Book Problem: B.7}
  \begin{enumerate}
    \item[a:] \textit{Give the truth table for the CMOS circuit in the Figure PB.5.}

    \item[b:] \textit{Derive the simplest sum-of-products expression for the truth table in (a). How many transistors are needed to build the sum-of-products circuit using the CMOS AND, OR and NOT gates?}

  \end{enumerate}

  \newpage
  \item \textit{Given the following schematic from the Texas Instruments CMOS BCD to 7-segment decoder CD4511B, estimate the number of transistors this circuit needs. You cannot ignore the FlipFlops at the inputs to the circuit (the box that is labeled as a PTGN). These are toggle flip flops with preset/clear. Model these like the example from class.}

  

\end{enumerate}
\end{document}
