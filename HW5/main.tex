\documentclass[12pt]{article}
\setlength{\oddsidemargin}{0in}
\setlength{\evensidemargin}{0in}
\setlength{\textwidth}{6.5in}
\setlength{\parindent}{0in}
\setlength{\parskip}{\baselineskip}

\usepackage[letterpaper, portrait, margin=1in]{geometry}
\usepackage[usenames, dvipsnames, rgb]{xcolor}
\usepackage{amsmath,amsfonts,amssymb,circuitikz,pdfpages,tikz}

\usetikzlibrary{matrix,calc,circuits.logic.US}
\tikzstyle{branch}=[fill, shape=circle, minimum size=3pt, inner sep=0pt]

%isolated term
%#1 - Optional. Space between node and grouping line. Default=0
%#2 - node
%#3 - filling color
\newcommand{\implicantsol}[3][0]{
    \draw[rounded corners=3pt, fill=#3, opacity=0.3] ($(#2.north west)+(135:#1)$) rectangle ($(#2.south east)+(-45:#1)$);
    }


%internal group
%#1 - Optional. Space between node and grouping line. Default=0
%#2 - top left node
%#3 - bottom right node
%#4 - filling color
\newcommand{\implicant}[4][0]{
    \draw[rounded corners=3pt, fill=#4, opacity=0.3] ($(#2.north west)+(135:#1)$) rectangle ($(#3.south east)+(-45:#1)$);
    }

%group lateral borders
%#1 - Optional. Space between node and grouping line. Default=0
%#2 - top left node
%#3 - bottom right node
%#4 - filling color
\newcommand{\implicantcostats}[4][0]{
    \draw[rounded corners=3pt, fill=#4, opacity=0.3] ($(rf.east |- #2.north)+(90:#1)$)-| ($(#2.east)+(0:#1)$) |- ($(rf.east |- #3.south)+(-90:#1)$);
    \draw[rounded corners=3pt, fill=#4, opacity=0.3] ($(cf.west |- #2.north)+(90:#1)$) -| ($(#3.west)+(180:#1)$) |- ($(cf.west |- #3.south)+(-90:#1)$);
}

%group top-bottom borders
%#1 - Optional. Space between node and grouping line. Default=0
%#2 - top left node
%#3 - bottom right node
%#4 - filling color
\newcommand{\implicantdaltbaix}[4][0]{
    \draw[rounded corners=3pt, fill=#4, opacity=0.3] ($(cf.south -| #2.west)+(180:#1)$) |- ($(#2.south)+(-90:#1)$) -| ($(cf.south -| #3.east)+(0:#1)$);
    \draw[rounded corners=3pt, fill=#4, opacity=0.3] ($(rf.north -| #2.west)+(180:#1)$) |- ($(#3.north)+(90:#1)$) -| ($(rf.north -| #3.east)+(0:#1)$);
}

%group corners
%#1 - Optional. Space between node and grouping line. Default=0
%#2 - filling color
\newcommand{\implicantcantons}[2][0]{
    \draw[rounded corners=3pt, opacity=.3] ($(rf.east |- 0.south)+(-90:#1)$) -| ($(0.east |- cf.south)+(0:#1)$);
    \draw[rounded corners=3pt, opacity=.3] ($(rf.east |- 8.north)+(90:#1)$) -| ($(8.east |- rf.north)+(0:#1)$);
    \draw[rounded corners=3pt, opacity=.3] ($(cf.west |- 2.south)+(-90:#1)$) -| ($(2.west |- cf.south)+(180:#1)$);
    \draw[rounded corners=3pt, opacity=.3] ($(cf.west |- 10.north)+(90:#1)$) -| ($(10.west |- rf.north)+(180:#1)$);
    \fill[rounded corners=3pt, fill=#2, opacity=.3] ($(rf.east |- 0.south)+(-90:#1)$) -|  ($(0.east |- cf.south)+(0:#1)$) [sharp corners] ($(rf.east |- 0.south)+(-90:#1)$) |-  ($(0.east |- cf.south)+(0:#1)$) ;
    \fill[rounded corners=3pt, fill=#2, opacity=.3] ($(rf.east |- 8.north)+(90:#1)$) -| ($(8.east |- rf.north)+(0:#1)$) [sharp corners] ($(rf.east |- 8.north)+(90:#1)$) |- ($(8.east |- rf.north)+(0:#1)$) ;
    \fill[rounded corners=3pt, fill=#2, opacity=.3] ($(cf.west |- 2.south)+(-90:#1)$) -| ($(2.west |- cf.south)+(180:#1)$) [sharp corners]($(cf.west |- 2.south)+(-90:#1)$) |- ($(2.west |- cf.south)+(180:#1)$) ;
    \fill[rounded corners=3pt, fill=#2, opacity=.3] ($(cf.west |- 10.north)+(90:#1)$) -| ($(10.west |- rf.north)+(180:#1)$) [sharp corners] ($(cf.west |- 10.north)+(90:#1)$) |- ($(10.west |- rf.north)+(180:#1)$) ;
}

%Empty Karnaugh map 4x4
\newenvironment{Karnaugh}%
{
\begin{tikzpicture}[baseline=(current bounding box.north),scale=0.8]
\draw (0,0) grid (4,4);
\draw (0,4) -- node [pos=1,above right,anchor=south west] {$x_3x_4$} node [pos=0.7,below left,anchor=north east] {$x_1x_2$} ++(135:1);
%
\matrix (mapa) [matrix of nodes,
        column sep={0.8cm,between origins},
        row sep={0.8cm,between origins},
        every node/.style={minimum size=0.3mm},
        anchor=8.center,
        ampersand replacement=\&] at (0.5,0.5)
{
                       \& |(c00)| 00         \& |(c01)| 01         \& |(c11)| 11         \& |(c10)| 10         \& |(cf)| \phantom{00} \\
|(r00)| 00             \& |(0)|  \phantom{0} \& |(1)|  \phantom{0} \& |(3)|  \phantom{0} \& |(2)|  \phantom{0} \&                     \\
|(r01)| 01             \& |(4)|  \phantom{0} \& |(5)|  \phantom{0} \& |(7)|  \phantom{0} \& |(6)|  \phantom{0} \&                     \\
|(r11)| 11             \& |(12)| \phantom{0} \& |(13)| \phantom{0} \& |(15)| \phantom{0} \& |(14)| \phantom{0} \&                     \\
|(r10)| 10             \& |(8)|  \phantom{0} \& |(9)|  \phantom{0} \& |(11)| \phantom{0} \& |(10)| \phantom{0} \&                     \\
|(rf) | \phantom{00}   \&                    \&                    \&                    \&                    \&                     \\
};
}%
{
\end{tikzpicture}
}

%Empty Karnaugh map 4x4
\newenvironment{Karnaugh5}%
{
\begin{tikzpicture}[baseline=(current bounding box.north),scale=0.8]
\draw (0,0) grid (4,4);
\draw (0,4) -- node [pos=1,above right,anchor=south west] {$x_4x_5$} node [pos=0.7,below left,anchor=north east] {$x_2x_3$} ++(135:1);
%
\matrix (mapa) [matrix of nodes,
        column sep={0.8cm,between origins},
        row sep={0.8cm,between origins},
        every node/.style={minimum size=0.3mm},
        anchor=8.center,
        ampersand replacement=\&] at (0.5,0.5)
{
                       \& |(c00)| 00         \& |(c01)| 01         \& |(c11)| 11         \& |(c10)| 10         \& |(cf)| \phantom{00} \\
|(r00)| 00             \& |(0)|  \phantom{0} \& |(1)|  \phantom{0} \& |(3)|  \phantom{0} \& |(2)|  \phantom{0} \&                     \\
|(r01)| 01             \& |(4)|  \phantom{0} \& |(5)|  \phantom{0} \& |(7)|  \phantom{0} \& |(6)|  \phantom{0} \&                     \\
|(r11)| 11             \& |(12)| \phantom{0} \& |(13)| \phantom{0} \& |(15)| \phantom{0} \& |(14)| \phantom{0} \&                     \\
|(r10)| 10             \& |(8)|  \phantom{0} \& |(9)|  \phantom{0} \& |(11)| \phantom{0} \& |(10)| \phantom{0} \&                     \\
|(rf) | \phantom{00}   \&                    \&                    \&                    \&                    \&                     \\
};
}%
{
\end{tikzpicture}
}

%Empty Karnaugh map 2x4
\newenvironment{Karnaughvuit}%
{
\begin{tikzpicture}[baseline=(current bounding box.north),scale=0.8]
\draw (0,0) grid (4,2);
\draw (0,2) -- node [pos=1,above right,anchor=south west] {$x_2x_3$} node [pos=0.7,below left,anchor=north east] {$x_1$} ++(135:1);
%
\matrix (mapa) [matrix of nodes,
        column sep={0.8cm,between origins},
        row sep={0.8cm,between origins},
        every node/.style={minimum size=0.3mm},
        anchor=4.center,
        ampersand replacement=\&] at (0.5,0.5)
{
                      \& |(c00)| 00         \& |(c01)| 01         \& |(c11)| 11         \& |(c10)| 10         \& |(cf)| \phantom{00} \\
|(r00)| 0             \& |(0)|  \phantom{0} \& |(1)|  \phantom{0} \& |(3)|  \phantom{0} \& |(2)|  \phantom{0} \&                     \\
|(r01)| 1             \& |(4)|  \phantom{0} \& |(5)|  \phantom{0} \& |(7)|  \phantom{0} \& |(6)|  \phantom{0} \&                     \\
|(rf) | \phantom{00}  \&                    \&                    \&                    \&                    \&                     \\
};
}%
{
\end{tikzpicture}
}

%Empty Karnaugh map 2x2
\newenvironment{Karnaughquatre}%
{
\begin{tikzpicture}[baseline=(current bounding box.north),scale=0.8]
\draw (0,0) grid (2,2);
\draw (0,2) -- node [pos=0.7,above right,anchor=south west] {b} node [pos=0.7,below left,anchor=north east] {a} ++(135:1);
%
\matrix (mapa) [matrix of nodes,
        column sep={0.8cm,between origins},
        row sep={0.8cm,between origins},
        every node/.style={minimum size=0.3mm},
        anchor=2.center,
        ampersand replacement=\&] at (0.5,0.5)
{
          \& |(c00)| 0          \& |(c01)| 1  \\
|(r00)| 0 \& |(0)|  \phantom{0} \& |(1)|  \phantom{0} \\
|(r01)| 1 \& |(2)|  \phantom{0} \& |(3)|  \phantom{0} \\
};
}%
{
\end{tikzpicture}
}

%Defines 8 or 16 values (0,1,X)
\newcommand{\contingut}[1]{%
\foreach \x [count=\xi from 0]  in {#1}
     \path (\xi) node {\x};
}

%Places 1 in listed positions
\newcommand{\minterms}[1]{%
    \foreach \x in {#1}
        \path (\x) node {1};
}

%Places 0 in listed positions
\newcommand{\maxterms}[1]{%
    \foreach \x in {#1}
        \path (\x) node {0};
}

\newcommand{\overbar}[1]{\mkern 1.5mu\overline{\mkern-1.5mu#1\mkern-1.5mu}\mkern 1.5mu}

\begin{document}
\title{Digital Logic Homework 4}

ECEN 2350 Spring 2017 \hfill Homework 4\\
Samuel Cuthbertson

\hrulefill

\begin{enumerate}
	\vspace{-4mm}
	\item \textit{Book Problems: 3.2, 3.3, 3.5 and 3.5}
	      \begin{enumerate}

	      	\item[(3.2)] \textit{Determine the decimal values of the following decimal values of the following one's complement numbers.}
	      	  \item \textit{0111011110} \\
	      	      First, note the that given number is positive (MSB is 0) and thus it's value is simply given by the below formula:
	      	      \begin{align*}
	      	      	V(b) &= b_0 * 2^0 + b_1 * 2^1 ... b_i * 2^i \\
                       &= 0*2^0 + 1*2^1 + 1*2^2 + 1*2^3 + 1*2^4 + 0*2^5 + 1*2^6 + 1*2^7 + 1*2^8 \\
                       &= \boxed{478}
	      	      \end{align*}

	         \item \textit{1011100111} \\
                Now, note that the value is negative (MSB is 1) and thus the magnitude will be given by examining it's complement (0100011000) with the formula mentioned in the previous problem.
                \begin{align*}
                  V(b) &= 0*2^0 + 0*2^1 + 0*2^2 + 1*2^3 + 1*2^4 + 0*2^5 + 0*2^6 + 0*2^7 + 1*2^8 \\
                       &= 280
                \end{align*}
                Thus, our value is \boxed{-280}
	      	  \item \textit{1111111110} \\
                Again, note the the value is negative and that the magnitude is given by it's complement (0000000001).
                \begin{align*}
                  V(b) &= 1*2^0 \\
                       &= 1
                \end{align*}
                Thus, our value is \boxed{-1}

          \vspace{10mm}
	      	\item[(3.3)] \textit{Determine the values of the following 2's complement numbers.} \\
                First, recall the below equations for the value of 2's complements numbers.
                \[
                V(b) = b_0 * 2^0 + b_1 * 2^1 + ... + b_{i-1} * 2^{i-1} - b_i * 2^i
                \]
            \setcounter{enumii}{0}
            \item \textit{0111011110}
              \begin{align*}
                V(b) &= 1*2^1 + 1*2^2 + 1*2^3 + 1*2^4 + 1*2^6 + 1*2^7 + 1*2^8 \\
                V(b) &= \boxed{478}
              \end{align*}
            \item \textit{1011100111}
              \begin{align*}
                V(b) &= 1*2^0 + 1*2^1 + 1*2^2 + 1*2^5 + 1*2^6 + 1*2^7 - 1*2^9 \\
                V(b) &= \boxed{-281}
              \end{align*}

            \item \textit{1111111110}
              \begin{align*}
                V(b) &= 1*2^1 + 1*2^2 + 1*2^3 + 1*2^4 + 1*2^5 + 1*2^6 + 1*2^7 + 1*2^8 - 1*2^9 \\
                V(b) &= \boxed{-2}
              \end{align*}

          \vspace{10mm}
          \item[(3.4)] \textit{Convert 73, 1906, -95 and -1630 into signed 12-bit numbers in the following representations.}
            \setcounter{enumii}{0}
            \item \textit{Sign and Magnitude}
              \begin{align*}
                73 = 64 + 8 + 1 &= \boxed{(000001001001)_2} \\
                1906 = 1024 + 512 + 256 + 64 + 32 + 16 + 2 &= \boxed{(011101110010)_2} \\
                \text{Note: \;} 95 = 64 + 16 + 8 + 4 + 2 + 1 = (000001011111)_2 \\
                \implies -95 &= \boxed{(100001011111)_2} \\
                \text{Note: \;} 1630 = 1024 + 512 + 64 + 16 + 8 + 4 + 2 = (01&1001011110)_2 &\\
                \implies -1630 &= \boxed{(111001011110)_2}
              \end{align*}
            \item \textit{1's Complement} \\
              Take the magnitudes from above, and for the negatives bitwise complement them.
              \begin{align*}
                73 &= \boxed{(000001001001)_2} \\
                1906 &= \boxed{(011101110010)_2} \\
                -95 &= \boxed{(111110100000)_2} \\
                -1630 &= \boxed{(100110100001)_2}
              \end{align*}
            \item \textit{2's Complement} \\
              Take the answers from immediately above, and for the negatives add one.
              \begin{align*}
                73 &= \boxed{(000001001001)_2} \\
                1906 &= \boxed{(011101110010)_2} \\
                -95 &= \boxed{(111110100001)_2} \\
                -1630 &= \boxed{(100110100010)_2}
              \end{align*}

          \vspace{10mm}
          \item[(3.5)] \textit{Perform the following operations using 8-bit 2's complement numbers and indicate whether arithmetic overflow occurs.}
          \begin{center}
            \begin{tabular}{c c c c c c c c c c}
                \\
                & 0 & 0 & 0 & 0 & 1 & 0 & 0 && Carry\\
                & 0 & 0 & 1 & 1 & 0 & 1 & 1 & 0 & x\\
              + & 0 & 1 & 0 & 0 & 0 & 1 & 0 & 1 & y\\
                \hline
                & 0 & 1 & 1 & 1 & 1 & 0 & 1 & 1 & x+y
            \end{tabular}

            No Arithmetic Overflow, $54 + 69 = 123$.

            \begin{tabular}{c c c c c c c c c c}
                \\
              1 & 1 & 1 & 1 & 1 & 1 & 0 & 0 && Carry\\
                & 0 & 1 & 1 & 1 & 0 & 1 & 0 & 1 & x\\
              + & 1 & 1 & 0 & 1 & 1 & 1 & 1 & 0 & y\\
                \hline
                & 0 & 1 & 0 & 1 & 0 & 0 & 1 & 1 & x+y
            \end{tabular}

            No Arithmetic Overflow, $117 - 34 = 83$.

            \begin{tabular}{c c c c c c c c c c}
                \\
              1 & 1 & 1 & 1 & 1 & 0 & 0 & 0 && Carry\\
                & 1 & 1 & 0 & 1 & 1 & 1 & 1 & 1 & x\\
              + & 1 & 0 & 1 & 1 & 1 & 0 & 0 & 0 & y\\
                \hline
                & 1 & 0 & 0 & 1 & 0 & 1 & 1 & 1 & x+y
            \end{tabular}

            No Arithmetic Overflow, $-33-72 = -105$.

            \begin{tabular}{c c c c c c c c c c}
                \\
              0 & 0 & 0 & 0 & 1 & 0 & 1 & 1 && Borrow\\
                & 0 & 0 & 1 & 1 & 0 & 1 & 1 & 0 & x\\
              - & 0 & 0 & 1 & 0 & 1 & 0 & 1 & 1 & y\\
                \hline
                & 0 & 0 & 0 & 0 & 1 & 0 & 1 & 1 & x-y
            \end{tabular}

            No Arithmetic Overflow, $54-43 = 11$.

            \begin{tabular}{c c c c c c c c c c}
                \\
              1 & 0 & 0 & 0 & 0 & 1 & 1 & 0 && Borrow\\
                & 0 & 1 & 1 & 1 & 0 & 1 & 0 & 1 & x\\
              - & 1 & 1 & 0 & 1 & 0 & 1 & 1 & 0 & y\\
                \hline
                & 1 & 0 & 1 & 0 & 1 & 1 & 1 & 1 & x-y
            \end{tabular}

            Arithmetic Overflow, $117--42 = 117 + 42 = 158 \neq -81$.

            \begin{tabular}{c c c c c c c c c c}
                \\
              1 & 1 & 1 & 0 & 1 & 1 & 0 & 0 && Borrow\\
                & 1 & 1 & 0 & 1 & 0 & 0 & 1 & 1 & x\\
              - & 1 & 1 & 1 & 0 & 1 & 1 & 0 & 0 & y\\
                \hline
                & 1 & 1 & 1 & 0 & 0 & 1 & 1 & 1 & x-y
            \end{tabular}

            No Arithmetic Overflow, $-45- -20 = -45 + 20 = -25$.
          \end{center}
	      \end{enumerate}

        \newpage
        \item \textit{Multiply the following numbers like in the manner shown in Figure 3.36 of the textbook.}
        \begin{enumerate}
          \item \textit{20*15} \\
            Note: $20 = 16 + 4 = (010100)_2$ and $15 = 8 + 4 + 2 + 1 = (001111)_2$
            \begin{center}
              \begin{tabular}{c c c c c c c c c c c c | c}
                &&&&&& 0 & 1 & 0 & 1 & 0 & 0 & $20$\\
                &&&&&$\times$ & 0 & 0 & 1 & 1 & 1 & 1 & $15$\\
                \hline
                &&& 1 & 1 & 1 & 1 & &&&&& Carry \\
                &&&&&& 0 & 1 & 0 & 1 & 0 & 0 & $P_0$ \\
                &&&&& 0 & 1 & 0 & 1 & 0 & 0 && $P_1$ \\
                &&&& 0 & 1 & 0 & 1 & 0 & 0 &&& $P_2$ \\
                &&& 0 & 1 & 0 & 1 & 0 & 0 &&&& $P_3$ \\
                && 0 & 0 & 0 & 0 & 0 & 0 &&&&& $P_4$ \\
                + & 0 & 0 & 0 & 0 & 0 & 0 &&&&&& $P_5$ \\
                \hline
                & 0 & 0 & 1 & 0 & 0 & 1 & 0 & 1 & 1 & 0 & 0 & $20*15$
              \end{tabular}
              \begin{align*}
                20*15 &= 300 \\
                (010100)_2 * (001111)_2 &= \boxed{(0100101100)_2}
              \end{align*}
            \end{center}

          \item \textit{-7*10} \\
            Note: $-7 = -8 + 1 = (11001)_2$ and $10 = 8 + 2 = (01010)_2$
            \begin{center}
              \begin{tabular}{c c c c c c c c c c c | c}
                &&&&&& 1 & 1 & 0 & 0 & 1 & $-7$\\
                &&&&&$\times$ & 0 & 1 & 0 & 1 & 0 & $10$\\
                \hline
                & 1 & 1 & 1 &&&&&&&& Carry \\
                &&&&&& 0 & 0 & 0 & 0 & 0 & $P_0$ \\
                & 1 & 1 & 1 & 1 & 1 & 1 & 0 & 0 & 1 && $P_1$ \\
                &&&& 0 & 0 & 0 & 0 & 0 &&& $P_2$ \\
                & 1 & 1 & 1 & 1 & 0 & 0 & 1 &&&& $P_3$ \\
              + && 0 & 0 & 0 & 0 & 0 &&&&& $P_4$ \\
                \hline
                ... & 1 & 1 & 1 & 0 & 1 & 1 & 1 & 0 & 1 & 0 & $-7*10$
              \end{tabular}
              \begin{align*}
                -7*10 &= -70 \\
                (11001)_2 * (01010)_2 &= \boxed{(10111010)_2}
              \end{align*}
            \end{center}

          \item \textit{-3*-12}\\
            Note that $-3 * -12 = 3*12$, and $3 = (00011)_2$ while $12 = (01100)_2$
            \begin{center}
              \begin{tabular}{c c c c c c c c c c c | c}
                &&&&&& 0 & 0 & 0 & 1 & 1 & $3$\\
                &&&&&$\times$ & 0 & 1 & 1 & 0 & 0 & $12$\\
                \hline
                &&&&& 1 & 1 &&&&& Carry \\
                &&&&&& 0 & 0 & 0 & 0 & 0 & $P_0$ \\
                &&&&& 0 & 0 & 0 & 0 & 0 && $P_1$ \\
                &&&& 0 & 0 & 0 & 1 & 1 &&& $P_2$ \\
                &&& 0 & 0 & 0 & 1 & 1 &&&& $P_3$ \\
              + && 0 & 0 & 0 & 0 & 0 &&&&& $P_4$ \\
                \hline
                && 0 & 0 & 0 & 1 & 0 & 0 & 1 & 0 & 0 & $3*12$
              \end{tabular}
              \begin{align*}
                3*12 &= 36 \\
                (00011)_2 * (01100)_2 &= \boxed{(0100100)_2}
              \end{align*}
            \end{center}
        \end{enumerate}
      \end{enumerate}
\end{document}
