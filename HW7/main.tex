\documentclass[12pt]{article}
\setlength{\oddsidemargin}{0in}
\setlength{\evensidemargin}{0in}
\setlength{\textwidth}{6.5in}
\setlength{\parindent}{0in}
\setlength{\parskip}{\baselineskip}

\usepackage[letterpaper, portrait, margin=1in]{geometry}
\usepackage[usenames, dvipsnames, rgb]{xcolor}
\usepackage{amsmath,amsfonts,amssymb,circuitikz,pdfpages,tikz,fancyvrb}
\usepackage{tikz-timing}

\usetikzlibrary{matrix,calc,circuits.logic.US, shapes.geometric}
\tikzstyle{branch}=[fill, shape=circle, minimum size=3pt, inner sep=0pt]
\tikzstyle{mux} = [ trapezium,   draw,
                    shape border rotate = 270, trapezium angle = 60,
                    inner ysep=0pt, outer sep=1pt, inner xsep=1pt,
                    text width = 1.5em, align = center,
                    node distance=3cm]


\newcommand{\overbar}[1]
    {\mkern1.5mu\overline{\mkern-1.5mu#1\mkern-1.5mu}\mkern1.5mu}

\begin{document}

\renewcommand{\arraystretch}{1.25}

\makeatletter
    \pgfdeclareshape{JK}{
      \savedanchor\northeast{%
        \pgfmathsetlength\pgf@x{\pgfshapeminwidth}%
        \pgfmathsetlength\pgf@y{\pgfshapeminheight}%
        \pgf@x=0.5\pgf@x
        \pgf@y=0.5\pgf@y
      }
      % This is redundant, but makes some things easier:
      \savedanchor\southwest{%
        \pgfmathsetlength\pgf@x{\pgfshapeminwidth}%
        \pgfmathsetlength\pgf@y{\pgfshapeminheight}%
        \pgf@x=-0.5\pgf@x
        \pgf@y=-0.5\pgf@y
      }
      \inheritsavedanchors[from=rectangle]
      \inheritanchorborder[from=rectangle]
      \anchor{center}{\pgfpointorigin}
      \anchor{north}{\northeast \pgf@x=0pt}
      \anchor{east}{\northeast \pgf@y=0pt}
      \anchor{south}{\southwest \pgf@x=0pt}
      \anchor{west}{\southwest \pgf@y=0pt}
      \anchor{north east}{\northeast}
      \anchor{north west}{\northeast \pgf@x=-\pgf@x}
      \anchor{south west}{\southwest}
      \anchor{south east}{\southwest \pgf@x=-\pgf@x}
      \anchor{J}{
        \pgf@process{\northeast}%
        \pgf@x=-1\pgf@x%
        \pgf@y=.5\pgf@y%
      }
      \anchor{CLK}{
        \pgf@process{\northeast}%
        \pgf@x=-1\pgf@x%
        \pgf@y=0\pgf@y%
      }
      \anchor{K}{
        \pgf@process{\northeast}%
        \pgf@x=-1\pgf@x%
        \pgf@y=-0.5\pgf@y%
      }
      \anchor{Q}{
        \pgf@process{\northeast}%
        \pgf@y=.5\pgf@y%
      }
      \anchor{Qn}{
        \pgf@process{\northeast}%
        \pgf@y=-.5\pgf@y%
      }

      \backgroundpath{
        % Rectangle box
        \pgfpathrectanglecorners{\southwest}{\northeast}
        % Angle (>) for clock input
        \pgf@anchor@JK@CLK
        \pgf@xa=\pgf@x \pgf@ya=\pgf@y
        \pgf@xb=\pgf@x \pgf@yb=\pgf@y
        \pgf@xc=\pgf@x \pgf@yc=\pgf@y
        \pgfmathsetlength\pgf@x{1.6ex} % size depends on font size
        \advance\pgf@ya by \pgf@x
        \advance\pgf@xb by \pgf@x
        \advance\pgf@yc by -\pgf@x
        \pgfpathmoveto{\pgfpoint{\pgf@xa}{\pgf@ya}}
        \pgfpathlineto{\pgfpoint{\pgf@xb}{\pgf@yb}}
        \pgfpathlineto{\pgfpoint{\pgf@xc}{\pgf@yc}}
        \pgfclosepath

        \begingroup

        \pgf@anchor@JK@K
        \pgftext[left,base,at={\pgfpoint{\pgf@x}{\pgf@y}},
          x=\pgfshapeinnerxsep]{\raisebox{-0.75ex}{K}}


        \pgf@anchor@JK@J
        \pgftext[left,base,at={\pgfpoint{\pgf@x}{\pgf@y}},
          x=\pgfshapeinnerxsep]{\raisebox{-0.75ex}{J}}

        \pgf@anchor@JK@Q
        \pgftext[right,base,at={\pgfpoint{\pgf@x}{\pgf@y}},
          x=-\pgfshapeinnerxsep]{\raisebox{-.75ex}{Q}}

        \pgf@anchor@JK@Qn
        \pgftext[right,base,at={\pgfpoint{\pgf@x}{\pgf@y}},
          x=-\pgfshapeinnerxsep]{\raisebox{-.75ex}{$\overline{\mbox{Q}}$}}

        \endgroup
      }
    }
    % Key to add font macros to the current font
    \tikzset{add font/.code={\expandafter\def\expandafter\tikz@textfont
            \expandafter{\tikz@textfont#1}}}

    % Define default style for this node
    \tikzset{flip flop/port labels/.style={font=\sffamily\scriptsize}}
    \tikzset{every JK node/.style={draw,minimum width=1.5cm,
            minimum height=2.25cm,inner sep=1mm,outer sep=0pt,
            cap=round,add font=\sffamily}}
\makeatother

\makeatletter
    \pgfdeclareshape{tff}{
      \savedanchor\northeast{%
        \pgfmathsetlength\pgf@x{\pgfshapeminwidth}%
        \pgfmathsetlength\pgf@y{\pgfshapeminheight}%
        \pgf@x=0.5\pgf@x
        \pgf@y=0.5\pgf@y
      }
      % This is redundant, but makes some things easier:
      \savedanchor\southwest{%
        \pgfmathsetlength\pgf@x{\pgfshapeminwidth}%
        \pgfmathsetlength\pgf@y{\pgfshapeminheight}%
        \pgf@x=-0.5\pgf@x
        \pgf@y=-0.5\pgf@y
      }
      \inheritsavedanchors[from=rectangle]
      \inheritanchorborder[from=rectangle]
      \anchor{center}{\pgfpointorigin}
      \anchor{north}{\northeast \pgf@x=0pt}
      \anchor{east}{\northeast \pgf@y=0pt}
      \anchor{south}{\southwest \pgf@x=0pt}
      \anchor{west}{\southwest \pgf@y=0pt}
      \anchor{north east}{\northeast}
      \anchor{north west}{\northeast \pgf@x=-\pgf@x}
      \anchor{south west}{\southwest}
      \anchor{south east}{\southwest \pgf@x=-\pgf@x}
      \anchor{T}{
        \pgf@process{\northeast}%
        \pgf@x=-1\pgf@x%
        \pgf@y=.5\pgf@y%
      }
      \anchor{CLK}{
        \pgf@process{\northeast}%
        \pgf@x=-1\pgf@x%
        \pgf@y=-.5\pgf@y%
      }
      \anchor{Q}{
        \pgf@process{\northeast}%
        \pgf@y=.5\pgf@y%
      }
      \anchor{Qn}{
        \pgf@process{\northeast}%
        \pgf@y=-.5\pgf@y%
      }

      \backgroundpath{
        % Rectangle box
        \pgfpathrectanglecorners{\southwest}{\northeast}
        % Angle (>) for clock input
        \pgf@anchor@tff@CLK
        \pgf@xa=\pgf@x \pgf@ya=\pgf@y
        \pgf@xb=\pgf@x \pgf@yb=\pgf@y
        \pgf@xc=\pgf@x \pgf@yc=\pgf@y
        \pgfmathsetlength\pgf@x{1.6ex} % size depends on font size
        \advance\pgf@ya by \pgf@x
        \advance\pgf@xb by \pgf@x
        \advance\pgf@yc by -\pgf@x
        \pgfpathmoveto{\pgfpoint{\pgf@xa}{\pgf@ya}}
        \pgfpathlineto{\pgfpoint{\pgf@xb}{\pgf@yb}}
        \pgfpathlineto{\pgfpoint{\pgf@xc}{\pgf@yc}}
        \pgfclosepath

        \begingroup

        \pgf@anchor@tff@T
        \pgftext[left,base,at={\pgfpoint{\pgf@x}{\pgf@y}},
          x=\pgfshapeinnerxsep]{\raisebox{-0.75ex}{T}}

        \pgf@anchor@tff@Q
        \pgftext[right,base,at={\pgfpoint{\pgf@x}{\pgf@y}},
          x=-\pgfshapeinnerxsep]{\raisebox{-.75ex}{Q}}

        \pgf@anchor@tff@Qn
        \pgftext[right,base,at={\pgfpoint{\pgf@x}{\pgf@y}},
          x=-\pgfshapeinnerxsep]{\raisebox{-.75ex}{$\overline{\mbox{Q}}$}}

        \endgroup
      }
    }
    % Key to add font macros to the current font
    \tikzset{add font/.code={\expandafter\def\expandafter\tikz@textfont
            \expandafter{\tikz@textfont#1}}}

    % Define default style for this node
    \tikzset{flip flop/port labels/.style={font=\sffamily\scriptsize}}
    \tikzset{every tff node/.style={draw,minimum width=1.5cm,
            minimum height=2.25cm,inner sep=1mm,outer sep=0pt,
            cap=round,add font=\sffamily}}
\makeatother

\makeatletter
    \pgfdeclareshape{dff}{
      \savedanchor\northeast{%
        \pgfmathsetlength\pgf@x{\pgfshapeminwidth}%
        \pgfmathsetlength\pgf@y{\pgfshapeminheight}%
        \pgf@x=0.5\pgf@x
        \pgf@y=0.5\pgf@y
      }
      % This is redundant, but makes some things easier:
      \savedanchor\southwest{%
        \pgfmathsetlength\pgf@x{\pgfshapeminwidth}%
        \pgfmathsetlength\pgf@y{\pgfshapeminheight}%
        \pgf@x=-0.5\pgf@x
        \pgf@y=-0.5\pgf@y
      }
      \inheritsavedanchors[from=rectangle]
      \inheritanchorborder[from=rectangle]
      \anchor{center}{\pgfpointorigin}
      \anchor{north}{\northeast \pgf@x=0pt}
      \anchor{east}{\northeast \pgf@y=0pt}
      \anchor{south}{\southwest \pgf@x=0pt}
      \anchor{west}{\southwest \pgf@y=0pt}
      \anchor{north east}{\northeast}
      \anchor{north west}{\northeast \pgf@x=-\pgf@x}
      \anchor{south west}{\southwest}
      \anchor{south east}{\southwest \pgf@x=-\pgf@x}
      \anchor{D}{
        \pgf@process{\northeast}%
        \pgf@x=-1\pgf@x%
        \pgf@y=.5\pgf@y%
      }
      \anchor{CLK}{
        \pgf@process{\northeast}%
        \pgf@x=-1\pgf@x%
        \pgf@y=-.5\pgf@y%
      }
      \anchor{Q}{
        \pgf@process{\northeast}%
        \pgf@y=.5\pgf@y%
      }
      \anchor{Qn}{
        \pgf@process{\northeast}%
        \pgf@y=-.5\pgf@y%
      }

      \backgroundpath{
        % Rectangle box
        \pgfpathrectanglecorners{\southwest}{\northeast}
        % Angle (>) for clock input
        \pgf@anchor@dff@CLK
        \pgf@xa=\pgf@x \pgf@ya=\pgf@y
        \pgf@xb=\pgf@x \pgf@yb=\pgf@y
        \pgf@xc=\pgf@x \pgf@yc=\pgf@y
        \pgfmathsetlength\pgf@x{1.6ex} % size depends on font size
        \advance\pgf@ya by \pgf@x
        \advance\pgf@xb by \pgf@x
        \advance\pgf@yc by -\pgf@x
        \pgfpathmoveto{\pgfpoint{\pgf@xa}{\pgf@ya}}
        \pgfpathlineto{\pgfpoint{\pgf@xb}{\pgf@yb}}
        \pgfpathlineto{\pgfpoint{\pgf@xc}{\pgf@yc}}
        \pgfclosepath

        \begingroup

        \pgf@anchor@dff@D
        \pgftext[left,base,at={\pgfpoint{\pgf@x}{\pgf@y}},
          x=\pgfshapeinnerxsep]{\raisebox{-0.75ex}{D}}

        \pgf@anchor@dff@Q
        \pgftext[right,base,at={\pgfpoint{\pgf@x}{\pgf@y}},
          x=-\pgfshapeinnerxsep]{\raisebox{-.75ex}{Q}}

        \pgf@anchor@dff@Qn
        \pgftext[right,base,at={\pgfpoint{\pgf@x}{\pgf@y}},
          x=-\pgfshapeinnerxsep]{\raisebox{-.75ex}{$\overline{\mbox{Q}}$}}

        \endgroup
      }
    }
    % Key to add font macros to the current font
    \tikzset{add font/.code={\expandafter\def\expandafter\tikz@textfont
            \expandafter{\tikz@textfont#1}}}

    % Define default style for this node
    \tikzset{flip flop/port labels/.style={font=\sffamily\scriptsize}}
    \tikzset{every dff node/.style={draw,minimum width=1.5cm,
            minimum height=2.25cm,inner sep=1mm,outer sep=0pt,
            cap=round,add font=\sffamily}}
\makeatother

\title{Digital Logic Homework 7}

ECEN 2350 Spring 2017 \hfill Homework 7\\
Samuel Cuthbertson

\hrulefill{}
\begin{enumerate}
  \item \textit{Book Problems: 5.7, 5.10, 5.16, 5.25}
  \begin{enumerate}
    \item[5.7:] \textit{Show how a JK flip-flop can be implemented with a T
                        flip-flop and other gates.}

      \vspace{3mm}
      Note that the characteristic equations for a T flip-flip and for a JK
        flip-flop are both shown below.
      \begin{center}
        \begin{minipage}{0.4\textwidth}
          \begin{center}
            \begin{tabular}{c | c}
                  T & $Q(t+1)$ \\
                  \hline
                  0 & $Q(t)$ \\
                  1 & $\overbar{Q(t)}$
            \end{tabular}
          \end{center}
        \end{minipage}
        \hfill
        \begin{minipage}{0.4\textwidth}
          \begin{center}
            \begin{tabular}{c c | c}
                  J & K & $Q(t+1)$ \\
                  \hline
                  0 & 0 & $Q(t)$ \\
                  0 & 1 & 0 \\
                  1 & 0 & 1 \\
                  1 & 1 & $\overbar{Q(t)}$
            \end{tabular}
        \end{center}
        \end{minipage}
      \end{center}
      \vspace{3mm}
      Therefor, for $J = K$ or $J \otimes K$, we can use J as T. For $J\neq K$,
        we need to set $T = J \oplus Q(t)$, as that will set $T=1$ when
        $Q(t) \neq J$ and will toggle $Q(t+1)$. This is implemented below.
      \begin{center}
        \begin{tikzpicture}[circuit logic US]
          %Inputs
          \node (J) at (0,6) {J};
          \node (K) at (0,5) {K};
          \node (CLK) at (0,4.185) {CLK};

          %Flip-flop
          \node[shape=tff, inner sep=2ex, draw] (TFF) at (10,4.75) {};

          %Gates
          \node[xor gate US, draw, logic gate inputs=nn] (JxorK) at (2,5.5){};
          \node[not gate US, draw,logic gate inputs=n](nJxorK)at(3.5,4.924){};
          \node[and gate US, draw, logic gate inputs=nn] (nJand) at (5,5) {};
          \node[xor gate US, draw, logic gate inputs=nn](JxorQ)at(5,6.075){};
          \node[and gate US, draw, logic gate inputs=nn](Qand)at(6.5,5.575){};
          \node[or gate US, draw, logic gate inputs=nn] (Tor) at (8,5.25) {};

          %Output
          \node (Q) at ($(TFF.Q) + (1.5,0)$) {$Q$};
          \node (nQ) at ($(TFF.Qn) + (1.5,0)$) {$\overbar{Q}$};

          %Connections
          \draw (CLK) -- ($(CLK) + (2.25,0)$) |- (TFF.CLK);

          \draw (J) -- ++(0:1) node[branch](jb){};
          \draw (jb) |- (JxorK.input 1);
          \draw (K) -- ++(0:1) |- (JxorK.input 2);

          \draw (jb) -- ++(0:3) node[branch](jb2){};
          \draw (jb2) |- (nJand.input 1);

          \draw (nJxorK.output) -- (nJand.input 2);
          \draw (JxorK) -- ++(0:0.75) node[branch](jb3){} |- (nJxorK.input);

          \draw (jb2) -- (JxorQ.input 2);
          \draw (TFF.Q) -- ++(0:.5) node[branch](q){} -- ++(90:1.25) --
            ++(180:7.25) -- ++(270:.41) -- (JxorQ.input 1);

          \draw (JxorQ.output) -| ++(0:.25) |- (Qand.input 1);
          \draw (jb3) -- (Qand.input 2);

          \draw (Qand.output) -- ++(0:.25) |- (Tor.input 1);
          \draw (nJand.output) -- ++(0:1.75) |- (Tor.input 2);

          \draw (Tor.output) -- ++(0:0.9) |- (TFF.T);

          \draw (TFF.Q) -- (Q);
          \draw (TFF.Qn) -- (nQ);

        \end{tikzpicture}
      \end{center}

    \item[5.10:] \textit{Write the behavioral Verilog code for a JK flip-flop.}

      \begin{Verbatim}
module jk_flip_flop(J, K, CLK, Q, nQ)
  input reg J, K, CLK;
  output reg Q, nQ;

  always@(posedge CLK)
  begin
    case({J, K})
      {1'b0, 1'b1}: Q <= 1'b0;
      {1'b1, 1'b0}: Q <= 1'b1;
      {1'b1, 1'b1}: Q <= ~Q;
    endcase
    nQ = ~Q
  end
endmodule
      \end{Verbatim}

    \item[5.16:] \textit{Design a three bit up/down counter using D flip-flops,
                        using a control signal of $\overbar{UP}/DOWN$.}

      For an up counter, we can use the equations $D_0 = Q_0 \oplus 1$,
        $D_1 = Q_1 \oplus Q_0$, $D_2 = Q_2 \oplus Q_1 Q_0$. For a down counter,
        the equations are similar: $D_0 = Q_0 \oplus 1$,
        $D_1 =\overbar{Q_0} \oplus Q_1 $, $D_2 = \overbar{Q_1} \oplus Q_1 Q_2$.
        As the only diffrence between these is whether to use $Q_{n-1}$ or
        $\overbar{Q_{n-1}}$ in the \texttt{xor} with the previous \texttt{and},
        we can use a \texttt{mux} at each stage with our control signal as a
        select.

      \begin{center}
        \begin{tikzpicture}[scale=.75, transform shape]
          %Inputs
          \node (CLK) at (-3, 1) {CLK};
          \node (CTRL) at (-1.75, 5.5) {$\overbar{UP}/DOWN$};

          %Q0
          \node[shape=dff, draw] (D0) at (0,3) {};
          \node[mux, draw] (M0) at (2.25,3.25) {0\\1};
          \node[xor gate US, draw, logic gate inputs=nn] (X0) at (4,3.34) {};
          \node (Q0o) at (1.32,7) {$Q_0$};

          %Q0 connections
          \draw (CTRL) -- ++(0:4) node[branch] (Q0CLB) {} -- (M0);
          \draw (D0.Q) -- ++(0:.57) node [branch] (Q0) {} -- ++(0:.57);
          \draw (D0.Qn) -- ++(0:.6) node [branch] (nQ0) {} |- (M0.south west);
          \draw (nQ0) -- ++(0,-1) node[branch] (nQ02){} -- ++(-2.5,0) |- (D0.D);
          \draw (CLK) -- ++(1.5,0) node [branch] (Q0CLK) {} |- (D0.CLK);
          \draw (M0) -- (X0.input 2);
          \draw (Q0) -- ++(0,1) node[branch] (Q02) {} -- (Q0o);

          %Q1
          \node[shape=dff, draw] (D1) at (6,3) {};
          \node[and gate US, draw, logic gate inputs=nn] (A11) at (8,3.64) {};
          \node[and gate US, draw, logic gate inputs=nn] (A12) at (8,2.35) {};
          \node[mux, draw] (M1) at (9.5,3.25) {0\\1};
          \node[xor gate US, draw, logic gate inputs=nn] (X1) at (11,3.34) {};
          \node (Q1o) at (7.05,7) {$Q_1$};

          %Q1 connections
          \draw (Q0CLB) -- ++(0:7.25) -- (M1);
          \draw (X0.output) -- ++(.35,0) |- (D1.D);
          \draw (D1.Q) -- ++(0:.3) node[branch] (Q1) {} -- (A11.input 2);
          \draw (D1.Qn) -- (A12.input 1);
          \draw (A11.output) -- ++(0:.78);
          \draw (A12.output) -- ++(0:.4) -- ++(0,.5) -- ++(.4,0);
          \draw (Q0CLK) -- ++(6,0) node[branch] (Q1CLK) {} |- (D1.CLK);
          \draw (M1) -- (X1.input 2);
          \draw (nQ02) -- ++(0:6) |- (A12.input 2);
          \draw (Q02) -- ++(0:6) |- (A11.input 1);
          \draw (Q1) -- ++(0,1.5) node[branch] (Q12) {} -- (Q1o);
          \draw (Q12) -- ++(-4,0) |- (X0.input 1);

          %Q2
          \node[shape=dff, draw] (D2) at (13,3) {};
          \node (Q2o) at (14.25,7) {$Q_2$};

          %Q2 connections
          \draw (X1.output) -- ++(.35,0) |- (D2.D);
          \draw (Q1CLK) -- ++(7,0) |- (D2.CLK);
          \draw (D2.Q) -- ++(.5,0) -- ++(0,1) node[branch] (Q2) {} -- (Q2o);
          \draw (Q2) -- ++(-4,0) |- (X1.input 1);

        \end{tikzpicture}
      \end{center}

    \item[5.25:] \textit{Using the circuit shown in the book, complete the
                         below timing diagram.}

      As all gates have a 1ns delay, any changes to \texttt{A} from \texttt{D}
        will take 2ns, while changes from \texttt{CLK} will take 1ns. Similarly,
        changes from \texttt{D} to \texttt{Q} will propagate in 4ns and any
        changes to \texttt{Q} from \texttt{CLK} will take 2ns.
      \vspace{3mm}
      \begin{center}
        \begin{tikztimingtable}
          Clock & L L L L L H H H H H L L L L L H H H H H L L L L L H H H H H\\
          D &     L L H H L L H H H H H L L L L L L L L L L L H H H H H H H H\\
          A &     H H H H H H L L H H H H H H H H L L L L L H H H H H H H H H\\
          Q &     L L L L L L L L L L H H H H H H H H H H L L L L L L L H H H\\
        \extracode
          \tablerules
          \begin{pgfonlayer}{background}
            \foreach \l in {1,...,30}
              \vertlines[help lines]{\l};
          \end{pgfonlayer}
        \end{tikztimingtable}
      \end{center}

  \end{enumerate}

  \newpage
  \item \textit{Draw a 4-bit Universal shift register that can parallel load,
                shift left, shift right, and synchronously clear. Provide a
                characteristic table for the control operations.}

    Shown below is an adaption of the circuit shown in Figure 5.18 of the
      textbook, extending the multiplexer to a 4-to-1 mux.
    \begin{center}
      \begin{tikzpicture}[scale=.75, transform shape]
        %Inputs
        \node (CLK) at (-3, -2.5) {CLK};
        \node (CTRL) at (-3, 3) {CTRL};
        \node (SI) at (-3, .85) {Serial In};

        %q0
        \node[mux, draw] (M0) at (0,0.55) {00\\01\\10\\11};
        \node[shape=dff, draw] (D0) at (1.5,0) {};
        \node (Q0o) at (2.5,4) {$Q_0$};
        \node (Q0i) at (-.75,-3.5) {$P_0$};
        \draw (D0.D) -- ++(-.4,0);
        \node (00) at (-1,1.35) {$0$};
        \draw (00) -- ++(.65,0);
        \draw (SI) -- ++(2.65,0);
        \draw (D0.Q) -- ++(.25,0) node[branch] (Q0) {} -- ++(1.15,0);
        \draw (Q0) -- ++(0,-2.25) -- ++(-4,0) -- ++(0,2) -- ++(1.15,0);
        \draw (Q0i) -- ++(0,3.3) -- ++(.4,0);
        \draw (Q0) -- (Q0o);
        \draw (CTRL) -| (M0);
        \draw (CLK) -- ++(3.25,0) node[branch](CLK0){} |- (D0.CLK);

        %q1
        \node[mux, draw] (M1) at (4,0.3) {00\\01\\10\\11};
        \node[shape=dff, draw] (D1) at (5.5,-.25) {};
        \node (Q1o) at (6.5,4) {$Q_1$};
        \node (Q1i) at (3.35,-3.5) {$P_1$};
        \draw (D1.D) -- ++(-.4,0);
        \node (01) at (3,1.05) {$0$};
        \draw (01) -- ++(.65,0);
        \draw (D1.Q) -- ++(.25,0) node[branch] (Q1) {} -- ++(1.15,0);
        \draw (Q1) -- ++(0,-2.25) -- ++(-3.5,0) -- ++(0,2) -- ++(0.65,0);
        \draw (Q1i) -- ++(0,3) -- ++(.3,0);
        \draw (Q1) -- (Q1o);
        \draw (CTRL) -| (M1);
        \draw (CLK0) -- ++(4,0) node[branch](CLK1){} |- (D1.CLK);

        %q2
        \node[mux, draw] (M2) at (8,0.1) {00\\01\\10\\11};
        \node[shape=dff, draw] (D2) at (9.5,-.5) {};
        \node (Q2o) at (10.5,4) {$Q_2$};
        \node (Q2i) at (7.35,-3.5) {$P_2$};
        \draw (D2.D) -- ++(-.4,0);
        \node (02) at (7,0.75) {$0$};
        \draw (02) -- ++(.65,0);
        \draw (D2.Q) -- ++(.25,0) node[branch] (Q2) {} -- ++(1.15,0);
        \draw (Q2) -- ++(0,-2.25) -- ++(-3.5,0) -- ++(0,2) -- ++(0.65,0);
        \draw (Q2i) -- ++(0,2.8) -- ++(.3,0);
        \draw (Q2) -- (Q2o);
        \draw (CTRL) -| (M2);
        \draw (CLK1) -- ++(4,0) node[branch](CLK2){} |- (D2.CLK);

        %q3
        \node[mux, draw] (M3) at (12,-.15) {00\\01\\10\\11};
        \node[shape=dff, draw] (D3) at (13.5,-.75) {};
        \node (Q3o) at (14.5,4) {$Q_3$};
        \node (Q3i) at (11.35,-3.5) {$P_3$};
        \draw (D3.D) -- ++(-.4,0);
        \node (03) at (7,0.75) {$0$};
        \draw (03) -- ++(.65,0);
        \draw (D3.Q) -- ++(.25,0) node[branch] (Q3) {} -- (Q3o);
        \draw (Q3) -- ++(0,-2) -- ++(-3.5,0) -- ++(0,1.75) -- ++(0.65,0);
        \draw (Q3i) -- ++(0,2.6) -- ++(.3,0);
        \draw (CTRL) -| (M3);
        \draw (CLK2) -- ++(4,0) node[branch](CLK3){} |- (D3.CLK);
      \end{tikzpicture}
    \end{center}

    \begin{center}
      \begin{tabular}{c | c}
        CTRL & Behaviour \\
        \hline
        00 & Syncronous Clear: $Q(t+1) = 0$ \\
        01 & Right Shift \\
        10 & Left Shift \\
        11 & Parallel Load
      \end{tabular}
    \end{center}

  \item \textit{Implement a 3-bit up-counter using only one D flip-flop, one
                T flip-flop, one JK flip-flop and one AND gate. Assume all
                flip-flops are positive edge triggered. Show your circuit and
                a timing diagram.}

    You don't have a power source or a clock, so all of this is moot, but
      if you could find both of those then you could implement an up counter
      as shown below.

    As the first bit $Q_0$ changes with every posedge of clock, we can simply
      attach $\overbar{Q_0}$ to the input of the D and use that output for
      $Q_0$.

    Similarly, as $Q_1$ toggles between 1 and 0 every time
      $\overbar{Q_0}$ changes from 0 to 1 (posedge) we can attach 1 as the input
      to the T flip-flop and $\overbar{Q_0}$ to the clock, and then use the
      output from the T flip-flop as $Q_1$.

    Finally, as $Q_2$ toggles everytime that $\overbar{Q_0}\overbar{Q_1}$ goes
      0 to 1 (posedge), we can attach the output from that {\tt and} gate to
      both J and K, making the JK flip-flop behave like a T flip-flop. Then, we
      can use the output from the JK flip-flop as $Q_2$.

    All this put together is shown below:

    \begin{center}
      \begin{tikzpicture}
          \node[shape=dff] (d0) at (0,0) {};
          \node[shape=tff] (t0) at (5,0) {};
          \node[shape=JK] (jk) at (10,0) {};
          \node (CLK) at (-2.5,-0.56) {$CLK$};
          \draw (CLK) -- (d0.CLK);

          \node (q0) at (1.25,-3) {$Q_0$};
          \draw (d0.Q) -| (q0);

          \node (q1) at (6.25,-3) {$Q_1$};
          \draw (t0.Q) -| (q1);

          \node (q2) at (11.25,-3) {$Q_2$};
          \draw (jk.Q) -| (q2);

          \node (t1) at (3.5,0.57) {$1$};
          \draw (t1) -- (t0.T);

          \node (j1) at (8.5,0.57) {$1$};
          \draw (j1) -- (jk.J);
          \node (j2) at (8.5,-0.56) {$1$};
          \draw (j2) -- (jk.K);

          \draw (d0.Qn) -- (t0.CLK);
          \draw (d0.Qn) -- ++(1,0) node[branch]{} -- ++(0,-1)
            node[branch](q0b){} -- ++(-3,0) |- (d0.D);

          \node[and gate US, logic gate inputs=nn, draw] (and1) at (7,-.65) {};
          \draw (q0b) -- ++(4.25,0) |- (and1.input 2);
          \draw (t0.Qn) -- (and1.input 1);

          \draw (and1.output) -- ++(.5,0) |- (jk.CLK);
      \end{tikzpicture}
    \end{center}

  \item \textit{Write the behavioural Verilog for 5.16, calling the module
                \texttt{up\_down\_counter}.}

    \begin{Verbatim}
module up_down_counter(clk, ctrl, value);
  input clk;
  input ctrl;
  output reg [2:0] value;

  always@(posedge clk)
  begin
    case(ctrl)
      1'b0: value <= value - 1;
      1'b1: value <= value + 1;
    endcase
  end
endmodule
    \end{Verbatim}
\end{enumerate}
\end{document}
